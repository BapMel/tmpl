\documentclass[a4paper, 11pt]{report}
\usepackage[T1]{fontenc}
\usepackage[utf8]{inputenc}
\usepackage[francais]{babel}
\usepackage{amsmath}
\usepackage{url}

\NoAutoSpaceBeforeFDP

\newcommand{\tmpl}{\texttt{tmpl}}

\author{Baptiste Mélès}
\title{Introduction au langage TMPL}
\date{Avril 2008}

\begin{document}
\maketitle


\part{Introduction}

\chapter{Présentation générale}

Le {\bf TMPL} est un langage de programmation simulant le comportement
d'une machine de Turing. Les initiales TMPL signifient Turing Machine
Programming Language, et le sigle se prononce comme le mot anglais
\emph{temple}.

\par

Dans ce document, après avoir rapidement rappelé les principes de base
d'une machine de Turing, nous présenterons ce langage, puis les
principales options de l'interprète \tmpl{}, c'est-à-dire du
logiciel qui exécute les programmes TMPL\footnote{Il faut donc
  distinguer TMPL et \tmpl{}~: le premier désigne le langage, le
  second un interprète des programmes écrits dans ce langage. Quelqu'un
  pourrait tout à fait écrire un autre interprète qui s'appellerait,
  mettons, \texttt{contemplation}, et alors on utiliserait TMPL sans
  \tmpl{}.}.

\par

\section{Licence}

Écrit en Perl, l'interprète \texttt{tmpl} est un logiciel libre sous
licence~GPL (version~3 ou ultérieure). Vous pouvez donc le télécharger
librement et gratuitement\footnote{Mais si vous souhaitez absolument
  témoigner de votre enthousiasme, rien ne vous empêche d'envoyer une
  carte postale à Baptiste Mélès, 1 villa des Nymphéas, 75020 Paris,
  surtout si la carte représente des pingouins ou un beau paysage.}, et
même modifier le code source puis redistribuer votre version
personnalisée, à la seule condition que votre logiciel soit également
placé sous la licence~GPL.


\section{Téléchargement}

Vous pouvez télécharger l'interprète \texttt{tmpl} en vous rendant sur
le site \url{http://www.baptiste.meles.free.fr}, puis en suivant les
liens. Il vous suffit ensuite d'installer le programme \texttt{tmpl}
dans un répertoire de votre choix.

\par

Le seul prérequis, mais de taille, est d'\textbf{avoir installé Perl}
sur votre ordinateur.

\par

Si vous utilisez Linux ou un autre dérivé d'\textsc{unix}, l'endroit le
plus approprié pour installer le programme est sans doute le répertoire
\texttt{\$HOME/bin/}, où \texttt{\$HOME} désigne votre répertoire
personnel~; ou bien le répertoire \texttt{/usr/bin/} si vous avez des
droits d'administrateur (root) sur votre machine. Ces remarques valent
vraisemblablement pour les utilisateurs de Mac (Mac~OS~X ou supérieur).

\par

XXXX Et sous Windows ?



\chapter{La machine de Turing}

%Dans l'article «~On ~»

Une machine de Turing, telle qu'elle est décrite dans l'article
«~Théorie des nombres calculables, suivie d'une application au problème
de la décision~» (XXXX réfces précises), est composée des éléments
suivants~:


\begin{enumerate}
\item un ruban~; 
\item une tête de lecture~; 
\item une tête d'écriture~;
\item des états.
\end{enumerate}

\par

On peut effectuer exactement quatre opérations~:

\begin{enumerate}
\item lire le symbole écrit sous la tête de lecture~;
\item écrire sur le ruban à l'emplacement actuel de la tête de lecture~;
\item modifier l'état de la machine~;
\item déplacer la tête de lecture.
\end{enumerate}



\part{Le langage TMPL}


\chapter{Démarrer, écrire, terminer}

Un exemple simple valant tous les grands discours, voici votre premier
programme en tmpl. 

\par


\section{Votre premier programme TMPL}


\subsection{Écriture du programme}

Ouvrez votre éditeur de texte favori (Emacs, Vi, Bloc-Notes de Windows,
Wordpad...). Tapez la ligne suivante~:

\begin{verbatim}
START: > 1 :STOP
\end{verbatim}

Enregistrez maintenant cette ligne dans un fichier nommé
\texttt{ecrit1.tmpl}, dans un répertoire quelconque de votre
ordinateur. Nous ne saurions à ce propos trop vous recommander (XXXX
répétition ?) de créer un répertoire nommé par exemple
\texttt{tmpl-test}, dans lequel vous pourrez mener vos expérimentations,
créer vos premiers programmes, etc.

\par



\subsection{Exécution du programme}

Fermez maintenant votre éditeur de texte. Ouvrez un terminal. Exécutez
votre programme en tapant la commande

\begin{verbatim}
tmpl ecrit1.tmpl
\end{verbatim}

Si tout se passe bien\footnote{Dans le cas contraire, c'est sans doute
  que votre fichier \texttt{tmpl} n'est pas exécutable (voir
  \texttt{chmod}), ou qu'il ne se trouve pas dans votre \texttt{\$PATH},
  ou encore que vous ne vous trouvez pas dans le répertoire du fichier
  \texttt{ecrit1.tmpl}.  }, vous verrez simplement ceci~:

\begin{verbatim}
hylas bap ~ $ tmpl ecrit1.tmpl
1
hylas bap ~ $
\end{verbatim}

\par

Vous avez écrit un programme, certes modeste, dont la seule fonction est
d'écrire le symbole~1 sur le ruban de la machine de Turing.  Commentons
maintenant, pas à pas, ce premier programme.



\section{Description du programme}

\subsection{L'état de la machine}

Le programme \texttt{ecrit1.tmpl} commence par «~\verb+START:+~», mot
spécial du langage qui désigne l'état de départ de la machine. Un
programme doit contenir \emph{une fois et une seule}\footnote{À
  proprement parler, nous ne dirions pas avec la même sévérité les
  expressions «~une fois~» et «~une seule~». Que \texttt{START:} figure
  «~une fois~» au moins dans votre programme, c'est une nécessité
  absolue, à défaut de laquelle votre machine ne se mettra pas même en
  marche. Un programme sans \texttt{START:} équivaut donc strictement à
  un programme absolument vide, ou même inexistant. En revanche, vous
  pourriez tout à fait, si le cœur vous en dit, créer un programme
  comportant plusieurs fois l'instruction \texttt{START:}. Ce ne serait
  pas interdit, mais tout simplement vain~; car seule la première
  occurrence serait exécutée. Il est donc indispensable que
  \texttt{START:} figure «~une fois~» dans votre programme, et
  raisonnable qu'il n'y figure qu'«~une seule~».} l'instruction
\verb+START:+, pour la simple et bonne raison qu'il ne peut avoir qu'un
seul et même point de départ.

\par

Plus généralement, un mot suivi du signe deux-points (\texttt{:}) et
placé en début de ligne désigne un \textbf{état} de la machine ---~pour
adopter le lexique de Turing, c'est une
«~m-configuration~». \texttt{START:} est donc un état de la machine
analogue à tous les autres, à ceci près qu'il est le premier.

\par

\subsection{L'écriture sur le ruban}

Notre programme contient ensuite l'instruction \verb+> 1+, qui signifie
«~écrire le symbole~1~»\footnote{Ceux qui connaissent un peu
  \textsc{unix} reconnaîtront le symbole \texttt{>} qui sert à rediriger
  la sortie d'un programme, typiquement dans un fichier ---~donc, d'une
  certaine façon, à «~écrire~».}.  Si vous remplacez~1 par n'importe
quel autre caractère, la machine l'écrira. 

\par

La principale contrainte ---~nous verrons les autres plus tard~--- est
que ce caractère doit être unique. Vous pouvez écrire \verb+> 9+, mais
pas \verb+> 10+~: pour écrire 10, vous devrez écrire~1 puis~0, comme
nous le verrons plus loin.

\par



\subsection{Le changement d'état de la machine}

Enfin, la ligne de notre programme se termine par le mot
\texttt{:STOP}. Un mot précédé du signe deux-points et placé en fin de
ligne désigne un \textbf{changement d'état} de notre machine. En
l'occurrence, \texttt{STOP:} est un mot spécial du langage TMPL qui
désigne l'arrêt de la machine.

\par

Un programme standard contient une et une seule fois \texttt{START:}, et
au moins une fois l'instruction \texttt{:STOP}. Mais vous pouvez
parfaitement concevoir un programme sans \texttt{START:} ---~un
programme vide~---, aussi bien qu'un programme sans \texttt{:STOP}, tel
qu'une boucle infinie\footnote{Tous les programmes dépourvus de
  \texttt{:STOP} sont infinis, mais la réciproque n'est pas vraie~: un
  programme peut être infini pour la simple et bonne raison qu'il ne
  passe jamais par \texttt{:STOP}, quand bien même ce changement d'état
  figure sur l'une des lignes du code.}. 
\par

Vous avez donc réalisé votre premier programme en TMPL. Félicitations~!
N'hésitez pas, d'ores et déjà, à mener des expériences, à écrire de
petits programmes en modifiant ou en ôtant tel ou tel mot.

\par


\chapter{Se déplacer}


Jusqu'ici, votre usage de la machine de Turing est assez limité~: vous
pouvez seulement démarrer, écrire un caractère, et terminer. En d'autres
termes, vous ne pouvez manipuler qu'une seule case du ruban de la
machine. Passons maintenant à des programmes un peu plus complexes, et
voyons comment exécuter plusieurs instructions successives, et se
déplacer sur le ruban.

\par


\subsection{Se déplacer d'un caractère}

Enregistrez dans le fichier \texttt{123.tmpl} le programme suivant.

\begin{verbatim}
START: > 1 -> :suite
suite: > 2 -> :suite2
suite2: > 3 :STOP
\end{verbatim}

On reconnaît ici l'état \texttt{START:} et le changement d'état
\texttt{:STOP}, qui vous sont désormais familiers. Mais cette fois, on
ne passe pas directement de \texttt{START:} à \texttt{:STOP} au sein de
la même ligne d'instruction.

\par

Au début, notre machine est dans l'état \texttt{START:}. Elle exécute
donc la première ligne d'instruction, qui consiste à écrire le
symbole~1. 

\par

Vient ensuite, toujours sur la première ligne, le symbole \texttt{->},
composé d'un tiret et du signe «~plus grand que~». Il signifie «~se
déplacer d'une case vers la droite~»~; il imite en effet une flèche
tournée vers la droite. Comme vous pouvez vous en douter, pour se
déplacer vers la gauche, on utilise le symbole \texttt{<-}. Dans le
programme qui nous intéresse actuellement, nous nous déplaçons donc
d'une case vers la droite. Enfin, nous passons à l'état
\texttt{:suite}. Notre ruban présente l'état suivant~: $$1\_$$ où le
symbole «~\_~» désigne l'emplacement actuel du ruban.

\par

Notre machine étant à l'état \texttt{suite}, elle va lire chaque ligne
d'instruction pour voir s'il convient de l'exécuter. La première ligne
ne correspond pas, mais la deuxième, si~; la machine écrit donc~2 à
l'emplacement actuel, puis se déplace vers la droite d'une case, et
passe à l'état \texttt{suivant2}. Notre ruban ressemble maintenant à
ceci~: $$12\_$$ Enfin, on exécute la troisième ligne du programme, qui
écrit 3 sur le ruban, et la machine s'arrête. Le ruban porte donc la
suite de caractères $$123$$.

\par


\chapter{Ergonomie du TMPL}

Si les machines de Turing étaient faites pour être simples
d'utilisation, cela se saurait. Aussi tout programme en TMPL, pour peu
qu'il dépasse la dizaine de lignes, deviendrait vite illisible.


\section{Ordre des lignes d'instruction}

\subsection{Un ordre arbitraire}

L'ordre des lignes d'instruction dans un programme est, dans une large
mesure, arbitraire. On peut donc presque toujours permuter sans dommage
les lignes d'un programme. Le programme

\begin{verbatim}
START: > 1 -> :suite
suite: > 2 -> :suite2
suite2: > 3 :STOP
\end{verbatim}
est donc strictement équivalent au programme
\begin{verbatim}
START: > 1 -> :suite
suite2: > 3 :STOP
suite: > 2 -> :suite2
\end{verbatim}
qui est à son tour identique au programme
\begin{verbatim}
suite: > 2 -> :suite2
suite2: > 3 :STOP
START: > 1 -> :suite
\end{verbatim}
et au programme
\begin{verbatim}
suite2: > 3 :STOP
suite: > 2 -> :suite2
START: > 1 -> :suite
\end{verbatim}

\par

Vous pouvez donc écrire votre programme dans l'ordre que vous
voulez. Mais nous vous recommandons tout de même de suivre, dans la
mesure du possible, un ordre un peu logique, et ce pour deux raisons. La
première est simplement psychologique~: votre code sera bien plus
lisible, partant plus simple à déboguer. La seconde raison est en
revanche bel et bien technique~: en suivant un ordre logique, vous
éviterez les surprises ---~entendez par là~: les bogues~--- liées au
caractère glouton des lignes d'instruction. Précisons ce dernier point. 


\subsection{Des lignes gloutonnes}

Il existe un seul cas de figure où l'ordre des lignes ait une
importance~: celui où une ligne est «~engloutie~» par une autre. Prenons
le programme

\begin{verbatim}
START: > 1 -> :suite
suite: > 3 :STOP
suite: > 7 :STOP
\end{verbatim}

Ce programme écrit~1, avance d'une case, et fait passer la machine à
l'état \texttt{suite} (première ligne). Ensuite, la première ligne qui
corresponde à cet état est la deuxième~; c'est donc elle qui est
exécutée. Le ruban, lorsque la machine s'arrête, porte donc le
nombre~13. Si en revanche on permute l'ordre des deux dernières lignes
de ce programme, alors notre ruban portera le nombre~17.

\par

Les deux lignes sont en effet concurrentes, et la première dans l'ordre
du programme «~engloutit~» l'autre. C'est le seul cas dans lequel
l'ordre des lignes d'un programme ait un effet. Mais si aucun couple de
lignes de votre programme ne crée de concurrence, alors l'ordre est
rigoureusement arbitraire.

\par

Remarque~: l'analyse des lignes reprend toujours depuis le début du
programme. XXXX



\section{Espaces}

Vous pouvez insérer autant d'espaces et de tabulations que vous le
souhaitez au début, au sein ou à la fin d'une ligne. Le langage les
ignore purement et simplement. Non seulement ce n'est pas interdit, mais
nous ne saurions trop vous le recomander, tant un programme aéré gagne
en lisibilité. Qu'il vous suffise de comparer, même si pour l'instant
vous ne comprenez pas leur signification, le programme

\begin{verbatim}
START:<>1->:suite
suite:>2->:suite2
suite2:>3:STOP
\end{verbatim}

\noindent avec le programme suivant, qui produit rigoureusement le même
résultat~:

\begin{verbatim}
START:     <     > 1     ->    :suite
suite:     > 2           ->    :suite2
suite2:    > 3                 :STOP
\end{verbatim}

Comme vous pouvez le constater par vous-même, la structure du second
apparaît bien plus nettement que celle du premier.

\par

\section{Sauts de ligne}

Autant les espaces sont tolérés et encouragés, autant les sauts de ligne
sont interdits à l'intérieur d'une instruction. Le programme

\begin{verbatim}
START:     < 0      > 1
->         :STOP
\end{verbatim}

vous renverra tout simplement l'erreur suivante~:

\begin{verbatim}
hylas bap ~ $ tmpl saut_ligne.tmpl
Erreur ("saut_ligne.tmpl":1) : Format de ligne invalide.
\end{verbatim}

\section{Commentaires}

Commençons tout de suite par une instruction qui ne se trouve pas dans
la machine telle que Turing l'a décrite~: les commentaires. Afin de
rendre vos programmes plus ilsibles, vous pouvez en effet insérer des
remarques qui ne seront ni lues, ni exécutées. Pour cela, votre ligne
doit commencer par le caractère~\texttt{\#} (dièse). Exemple~:

\begin{verbatim}
# Ceci est un commentaire ; il ne sera pas lu par notre machine de 
# Turing.
\end{verbatim}

\par

Une ligne, absolument facultative, a un caractère particulier : le
shebang. XXXXX

XXXXX

Le langage TMPL contient une et une seule instruction pour chacune de
ces actions, chaque instruction devant occuper une et une seule ligne.
Une ligne typique d'instructions en TMPL présente l'aspect suivant~:

\begin{verbatim}
etat:       < 0      > 2     ->3      :suite
\end{verbatim}



\section{État de la machine}

Chaque ligne doit commencer par la mention d'un état de la machine, qui
est un nom quelconque, contenant un nombre quelconque de caractères
alphanumériques, et dont la fin est marquée par le caractère
deux-points. Exemples~: 

\begin{verbatim}
START:
a:
1729:
abc123:
\end{verbatim}

Parmi les noms d'état, un seul a un statut particulier, à savoir l'état
START, qui désigne l'état initial du programme.



\section{Changement d'état de la machine}

Toute ligne du programme doit également se terminer par un changement
d'état, chaîne alphanumérique quelconque \textbf{commençant} par le
caractère deux-points. Exemples~:


\begin{verbatim}
:STOP
:abc123
:1729
:a
\end{verbatim}

Un seul changement d'état a un statut particulier, à savoir STOP, qui
indique à la machine de s'arrêter. Le plus court programme terminant est
donc~: 


\begin{verbatim}
# Programme vide, qui se termine instantanément.
START: :STOP
\end{verbatim}

\textbf{Toute ligne du programme doit impérativement contenir un état et
  un changement d'état.}  Notez que les deux états ne sont pas
nécessairement distincts~: on peut donc parfaitement écrire des lignes
récursives comme les suivantes~:


\begin{verbatim}
# Ces lignes sont vides et constituent des boucles infinies.
abc123:  :abc123 
START:   :START
\end{verbatim}


\section{Écrire un symbole}

Pour que notre machine écrive un symbole, notre instruction doit
comporter, entre l'état et le changement d'état, le signe «~>~»
(«~plus grand que~») suivi du caractère qui nous intéresse. Voici par
exemple comment écrire exclusivement un~1~:



\begin{verbatim}
# Écrire 1 et terminer.
START: 	  >1    :STOP
\end{verbatim}


Remarquons tout de suite que le nombre d'espaces n'est pas pris en
compte par le langage TMPL. Les lignes suivantes sont donc
rigoureusement équivalentes~:



\begin{verbatim}
# Formulations équivalentes.
START:        >1       :STOP
START: >1 :STOP
START: > 1 :STOP
\end{verbatim}


Il est souvent très utile d'exploiter cette possibilité ouverte par le
langage, car vous vous apercevrez très rapidement que vos programmes
gagneront en lisibilité. Les tabulations vous permettent en effet
d'aligner verticalement toutes les instructions de même type.



\section{Lire un caractère}

Avant d'exécuter une ligne d'instructions, notre machine effectue deux
vérifications~: 



\begin{enumerate}
\item que son état soit bien celui qui est indiqué au début de la
  ligne~;
\item qu'elle lise bien sous la tête de lecture le caractère qu'on lui
  indique.
\end{enumerate}


Par exemple, prenons le programme suivant, qui écrit d'abord un~0, puis,
s'il lit un~0 (ce qui, convenons-en, sera nécessairement le cas), écrit
un~1 à la place~:



\begin{verbatim}
# J'écris d'abord 0.
START:         > 0   :boucle

# Puis, si je suis dans l'état "boucle" ET que je lis 0, alors j'écris 1.
boucle: < 0  > 1   :STOP 
\end{verbatim}


On peut donc réaliser le programme infini suivant, qui remplace un~0
par~1 et vice versa~:



\begin{verbatim}
START:       >0  :boucle
boucle:  <1  >0  :boucle
boucle:  <0  >1  :boucle
\end{verbatim}


Si l'on suit le déroulement de ce programme, nous voyons qu'il exécute
d'abord la ligne de l'état START, donc écrit~0. Il entre alors dans
l'état «~boucle~», comme indiqué à la fin de la même ligne. Il va donc
chercher si l'un des lignes correspondant à l'état «~boucle~» peut être
exécutée. La deuxième ligne de notre programme ne remplit pas les
conditions nécessaires, car elle ne peut être exécutée que si le
programme lit le symbole~1~; or, nous avons écrit, pour l'instant, le
symbole~0. La ligne suivante, en revanche, peut être exécutée, car nous
sommes bien dans l'état «~boucle~» et lisons bien le caractère~0. Nous
pouvons donc, comme le veut la troisième ligne du programme, écrire le
symbole~1. Ensuite, nous voyons que la deuxième ligne peut être
exécutée, puis la troisième, puis la deuxième, etc. à l'infini. Notre
machine écrira puis effacera des~0 et des~1 pour l'éternité.


< : lire un caractère non nul

non précisé (APRÈS les autres cas, sinon il est glouton et absorbe tout)
: toutes les autres possibilités




\section{Déplacer la tête}

La dernière commande du langage TMPL permet de déplacer la tête de
lecture et d'écriture. Pour cela, il faut taper une flèche, soit avec le
signe «~plus petit que~» suivi d'un tiret (<-), soit avec un tiret
suivi du signe «~plus grand que~» (->), le tout étant suivi d'un
nombre indiquant de combien de cases la tête doit se déplacer dans l'une
ou l'autre direction. Exemples~:



\begin{verbatim}
# Ce programme va écrire les chiffres 123 en commençant par le 2.
START:  >2   ->1   :nb2
nb2:    >3   <-2   :nb3
nb3:    >1            :STOP
\end{verbatim}



\part{L'interprète tmpl}

L'interprète \tmpl{} a pour principale fonction d'exécuter vos
programmes TMPL, mais ne s'y limite pas, tant s'en faut. Aussi
allons-nous vous présenter un panorama de ses fonctionnalités.

\par

\section{Version de l'interprète}

L'option \texttt{-v} affiche la version de \tmpl{} que vous
utilisez actuellement.

\begin{verbatim}
hylas bap ~ $ tmpl -v
tmpl 1.0
hylas bap ~ $
\end{verbatim}

\par

L'information est d'importance, car le langage TMPL devrait recevoir une
extension importante entre les versions~1.0 ---~dernière version en
date~--- et 2.0.



\section{Aide}

L'option \texttt{-h} affiche l'aide de \tmpl{}. C'est un pense-bête
pratique pour réviser les options les plus exotiques de notre
interprète.

\begin{verbatim}
hylas bap ~ $ tmpl -h
tmpl : interprète du langage TMPL (Turing Machine Programming Language).
Version 1.0

Syntaxe :
tmpl [-v|-h]
tmpl [-p|-t|-x|-d secondes|-s étapes] [prog1 prog2 ...]

Options
  -v            Version de tmpl
  -h            Cette aide
  -p            Attend que l'on appuie sur Entrée entre deux opérations
  -t            Affichage du programme sous forme de tableau
  -x            Exécution eXplicite du programme (numéro de ligne)
  -d secondes   Nombre (entier ou décimal) de secondes entre deux
                instructions
  -s étapes     Nombre (entier) d'instructions à exécuter
hylas bap ~ $
\end{verbatim}



\section{Tabularisation d'un programme}

L'option \texttt{-t} de \texttt{tmpl}, suivie d'un ou plusieurs noms de
programmes, est très préciseuse pour le débogage de programmes
TMPL. Elle en affiche en effet la structure sous forme de tableau. En
ceci, l'affichage se rapproche de celle que propose Turing dans son
article. 

\par

Prenons le programme suivant, conçu pour afficher la suite de
nombres $$0\ 1\ 0\ 1\ 0\ 1\ 0\ 1\ 0\ 1\ 0\ 1\ ...$$ Les lecteurs de
Turing auront vraisemblablement reconnu le premier programme proposé
dans la «~Théorie des nombres calculables~»\footnote{Plus précisément,
  il s'agit de la deuxième formulation du premier programme. XXXX
  réfces}. Nous l'enregistrons dans un fichier \texttt{nb\_entiers.tmpl}
(l'extension \emph{tmpl}, que nous adoptons par convention, est
absolument arbitraire~: elle n'a strictement aucune signification pour
notre interprète).

\begin{verbatim}
#!/home/bap/bin/tmpl

# Si je lis un 0, j'avance de 2 cases et écris 1.
START:                  :b
b:      <0      ->2     :b2
b2:         >1          :b

# Si je lis un 1, j'avance de deux cases et écris 0.
b:      <1      ->2     :b3
b3:         >0          :b

# Et si je ne lis rien du tout, alors j'écris 0.
b:      <   >0          :b
\end{verbatim}

Un programme

 à lire et à
comprendre au premier coup d'œil. Nous pouvons en construire une
représentation plus régulière, indépendante des partis pris stylistiques
du développeur (espaces, tabulations, commentaires, sauts de lignes...),
avec \texttt{tmpl~-t}~:

\begin{verbatim}
hylas bap ~ $ ./tmpl -t nb_entiers.tmpl
Ligne  État            Lect.  Écr.  Dépl.   Nouvel état
-------------------------------------------------------
4      START                                b
5      b               0            2       b2
6      b2                     1             b
9      b               1            2       b3
10     b3                     0             b
13     b               " "    0             b
hylas bap ~ $
\end{verbatim}


tmpl -e < foo.tmpl

équivaut à

tmpl foo.tmpl


\part{Annexes}

\section{Les programmes de Turing}

\section{Une machine universelle}

\section{Un Hello world}

\section{Messages d'erreur}

\section{Manuel technique}

Section destinée aux développeurs. 

La structure du programme tmpl.

Les principales structures de données, et surtout le ruban.


\end{document}
